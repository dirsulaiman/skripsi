
\chapter{HASIL DAN PEMBAHASAN}

% \section{}
% \section{Cara Implementasi}
% \section{Kinerja FPGA}


% cara implementasi
% perbandingan hasil

% \section{}
% \subsection{Citra Digital}

% video stream sebagai citra
% penerapan filter spasial
% - - - -
% Analisis Kinerja

\blindtext

\blindtext

\blindtext

\begin{atable}
    \caption{Perbandingan waktu komputasi}
    \label{table:hasil-fps}
    \csvreader[
        head to column names,
        tabular=lcc,
        before table=\rowcolors{2}{gray!15}{gray!30},
        table head= \rowcolor{gray!50!black} 
            \color{white} Filter & 
            \color{white} Tanpa FPGA & 
            \color{white} Dengan FPGA \\]
        {tables/hasil-fps.csv}
        {filter=\filter, tanpafpga=\tanpafpga, denganfpga=\denganfpga}
        {\filter & \tanpafpga & \denganfpga }
\end{atable}

\blindtext
