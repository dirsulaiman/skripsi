
\chapter{HASIL DAN PEMBAHASAN}


% Jelaskan Proses Implementasi 
\section{Implementasi pada FPGA Development Board}

Pada penelitian ini digunakan 6 kernel berbeda berukuran 3x3 untuk penerapan filter spasial linear pada video \textit{stream} 720p 60 FPS. Penerapan ini dilakukan pada FPGA Development Board dengan menggunakan prosesor ARM dan FPGA. Kemudian dilakukan analisis kinerja dari keduanya untuk menunjukkan masing-masing waktu komputasi, FPS, persentase penggunaan CPU, penggunaan \textit{memory}, \textit{resident memory} (RES), \textit{shared memory} (SHR), dan \textit{virtual memory} (VIRT) pada masing-masing kernel.

FPGA Development Board dirangkai seperti yang telah disebutkan pada Bab sebelumnya, pada gambar \ref{fig:rancangan-sistem}. HDMI Output pada FPGA dihubungkan ke monitor dan HDMI Input dihubungkan ke \textit{source} dalam hal ini laptop Lenovo Ideapad 320. FPGA Development Board juga dihubungkan ke \textit{router} menggunakan kabel UART agar dapat diakses menggunakan protokol \textit{ssh}. Selanjutnya memasang semua \textit{library} yang dibutuhkan untuk melakukan penerapan filter spasial linear pada video \textit{stream} menggunakan FPGA Development Board.

% \subsection{Representasi Video Stream sebagai Citra Digital}
% \subsection{Konversi Frame menjadi Grayscale}
\subsection{Penerapan Filter Spasial}
Setiap frame dari \textit{source} video \textit{stream} dibaca sebagai citra digital yang direpresentasikan sebagai matriks berukuran 1280x720 dengan rentang nilai keabuan pada masing-masing piksel yaitu dari 0 sampai 255. Setiap piksel pada matriks tersebut hanya memiliki satu lapis warna sebab citra yang diterima dari source adalah citra \textit{grayscale}, tidak seperti citra warna yang memiliki tiga lapis warna pada setiap pikselnya. 
% padding?

Proses filter spasial dilakukan dengan operasi konvolusi pada setiap matriks dengan kernel yang telah ditentukan sebelumnya. Operasi konvolusi ini menghasilkan matrix baru dengan ukuran 1280x720. Matriks hasil tersebut selanjutnya direpresentasikan kembali sebagai citra digital yang selanjutnya disebut sebagai hasil filter. Hasil filter dari setiap frame ini ditampilkan ke monitor melalui HDMI Output pada FPGA Development Board secara berkesinambungan sehingga tampak seperti video.

\begin{afigure}
    \includegraphics[width=0.8\linewidth, center]{images/output-image/input1-grayscale.png}
    \caption{Contoh Frame Grayscale.}
    \label{fig:input-grayscale}
\end{afigure}

Salah satu contoh frame dari \textit{source} dapat dilihat pada gambar \ref{fig:input-grayscale}. Selanjutnya dilakukan filter spasial menggunakan 6 kernel yang telah ditentukan sebelumnya.

\subsubsection{Average Blur}
Penerapan filter spasial pada frame \textit{grayscale} yang berukuran 1280x720 pixel dengan kernel \textit{average blur} (\ref{kernel:average}) yang berukuran 3x3 menghasilkan cita blur yang berukuran 1280x720. Hasil filter \textit{average blur} dapat dilihat pada gambar \ref{fig:output-averageblur}. Filter seperti ini dapat digunakan untuk mengurangi derau pada citra.
\begin{afigure}
    \includegraphics[width=0.8\linewidth, center]{images/output-image/input1-averageblur.png}
    \caption{Hasil filter Average Blur.}
    \label{fig:output-averageblur}
\end{afigure}

\subsubsection{Gaussian Blur}
Penerapan filter spasial dengan kernel \textit{gaussian blur} (\ref{kernel:gaussianblur}) yang berukuran 3x3 menghasilkan cita blur yang secara kasat mata mirip dengan filter \textit{average blur}. Namun apabila diperhatikan nilai masing-masing pixel pada gambar \ref{fig:output-gaussianblur} akan terlihat perbedaan dengan nilai masing-masing pixel pada gambar \ref{fig:output-averageblur}. Hal ini disebabkan oleh nilai bobot pada kernel \textit{gaussian blur} yang berbeda dengan kernel \textit{average blur} sehingga hasil konvolusinya juga berbeda. 
\begin{afigure}
    \includegraphics[width=0.8\linewidth, center]{images/output-image/input1-gaussianblur.png}
    \caption{Hasil filter Gaussian Blur.}
    \label{fig:output-gaussianblur}
\end{afigure}

\subsubsection{Laplacian}
Penerapan filter spasial dengan kernel \textit{laplacian} (\ref{kernel:laplacian}) menghasilkan cita biner yang hanya direpresentasikan dengan warna hitam dan putih saja, dapat dilihat pada gambar \ref{fig:output-laplacian}. Filter seperti ini dapat digunakan pada metode deteksi tepi dalam proses pengolahan citra digital.
\begin{afigure}
    \includegraphics[width=0.8\linewidth, center]{images/output-image/input1-laplacian.png}
    \caption{Hasil filter Laplacian.}
    \label{fig:output-laplacian}
\end{afigure}

\subsubsection{Sharpening}
Penerapan filter spasial dengan kernel \textit{sharpening} (\ref{kernel:sharpen}) dapat meningkatkan detail (seperti garis) pada citra, namun dapat juga dapat menimbulkan derau pada citra apabila bobot kernelnya tidak sesuai. Filter seperti ini lebih tepat digunakan untuk memperbaiki kualitas citra (dengan nilai kernel yang sesuai). Hasil filter \textit{sharpening} ini dapat dilihat pada gambar \ref{fig:output-sharpen}.
\begin{afigure}
    \includegraphics[width=0.8\linewidth, center]{images/output-image/input1-sharpen.png}
    \caption{Hasil filter Sharpening.}
    \label{fig:output-sharpen}
\end{afigure}

\subsubsection{Sobel Horizontal}
Penerapan filter spasial dengan kernel \textit{sobel horizontal} (\ref{kernel:sobel}) menghasilkan cita biner, dapat dilihat pada gambar \ref{fig:output-sobelhor}. Filter seperti lebih tepat digunakan pada metode deteksi tepi di citra yang banyak mengandung garis horizontal.
\begin{afigure}
    \includegraphics[width=0.8\linewidth, center]{images/output-image/input1-sobelhor.png}
    \caption{Hasil filter Sobel Horizontal.}
    \label{fig:output-sobelhor}
\end{afigure}

\subsubsection{Sobel Vertical}
Penerapan filter spasial dengan kernel \textit{sobel vertical} (\ref{kernel:sobel}) menghasilkan cita biner, dapat dilihat pada gambar \ref{fig:output-sobelver}. Sama halnya dengan filter \textit{sobel horizontal}, filter \textit{sobel vertical} juga dapat digunakan untuk metode deteksi tepi, terutama pada citra yang banyak mengandung garis verikal.
\begin{afigure}
    \includegraphics[width=0.8\linewidth, center]{images/output-image/input1-sobelver.png}
    \caption{Hasil filter Sobel Vertical.}
    \label{fig:output-sobelver}
\end{afigure}

\subsection{Penerapan Filter Spasial menggunakan Prosesor ARM dan FPGA}
Penerapan filter spasial menggunakan prosesor ARM dilakukan dengan menggunakan \textit{library} OpenCV dengan bahasa pemrograman Python yang dijalankan pada FPGA Development Board. Sedangkan untuk penerapan filter spasial menggunakan FPGA dilakukan dengan menggunakan \textit{library} xfOpenCV dari Xilinx. \textit{Library} xfOpenCV ini adalah library khusus yang dimodifikasi dari OpenCV sehingga proses komputasinya dapat dilakukan dengan FPGA, bukan dengan prosesor ARM yang pada FPGA Development Board ini. Lebih lanjut mengenai program yang digunakan dapat dilihat pada lampiran \textit{pynq-filter-spasial-arm} (\ref{code:filter-spasial-FPGA}) dan \textit{pynq-filter-spasial-fpga} (\ref{code:filter-spasial-ARM}).

\subsection{Proses Evaluasi Kinerja}
Pada penelitian ini proses dalam evaluasi kinerja dilakukan dalam dua tahap. Proses pertama untuk menghitung waktu komputasi dan FPS dari masing-masing kernel dengan prosesor ARM dan FPGA. Kemudian proses kedua untuk mencatat persentase penggunaan CPU dan penggunaan memory termasuk \textit{resident memory}, \textit{shared memory} dan \textit{virtual memory}. 

\subsubsection{Menghitung Waktu Komutasi dan FPS}
Pada proses ini peneliti melakukan percobaan dengan menggunakan 50 frame dan 200 frame dari \textit{source} untuk dihitung FPS dan waktu komputasinya. Percobaan dilakukan sebanyak 5 kali menggunakan 50 frame untuk masing-masing kernel pada prosesor ARM dan FPGA, dan 5 kali percobaan menggunakan 200 frame untuk masing-masing kernel. Hasil masing-masing percobaan ini dapat dilihat pada daftar lampiran (\ref{lampiran:waktu-komputasi}).

Menghitung waktu komputasi dilakukan dengan cara mencatat waktu mulai dan waktu semua frame selesai difilter pada setiap percobaan. Waktu komputasi diperoleh dari selisih antara waktu selesai dengan waktu mulai, seperti pada persamaan \ref{eq:time}. Dilakukan dengan 50 frame dan 200 frame pada masing-masing kernel dengan menggunakan prosesor ARM dan FPGA. Proses ini dilakukan dengan menggunakan \textit{library time} pada bahasa pemrograman Python, dapat dilihat pada gambar \ref{code:calculate-time}.
\begin{afigure}
    \lstinputlisting[frame=single, style=python]{images/programs/calculate-time.py}
    \caption{Menghitung waktu komputasi dengan library time di Python.}
    \label{code:calculate-time}
\end{afigure}

Hasil waktu komputasi ini dicatat dan kemudian digunakan untuk menghitung FPS dari masing-masing percobaan. Selanjutnya FPS masing-masing percobaan ini dihitung dengan menggunakan persamaan \ref{eq:fps}.

\subsubsection{Mencatat Penggunaan Memory dan CPU (Resource)}
Pada proses ini peneliti menggunakan fitur yang tersedia pada sistem operasi Linux yang berjalan di FPGA Development Board untuk membantu mencatat penggunaan CPU dan memory (\textit{resource}) pada saat proses penerapan filter spasial. Program ini menampilkan data tentang proses yang berjalan seperti ID sebuah proses, persentase memory yang digunakan oleh sebuah proses, persentase CPU yang digunakan, berapa lama sebuah proses berjalan, \textit{resident memory}, \textit{shared memory} dan \textit{virtual memory}. Contoh output program tersebut dapat dilihat pada gambar \ref{fig:top-linux}.
\begin{afigure}
    \includegraphics[width=0.8\linewidth, center]{images/programs/top-linux.png}
    \caption{Tampilan program \textbf{top}.}
    \label{fig:top-linux}
\end{afigure}

Peneliti melakukan 5 kali percobaan untuk mencatat penggunaan memory dan CPU dari masing-masing kernel dengan prosesor ARM dan FPGA. Percobaan ini dilakukan dengan menggunakan Jupyter Notebook yang berjalan pada FPGA Development Board yang dapat diakses menggunakan \textit{web browser} dari perangkat lain. Serta digunakan protokol SSH untuk mengakses FPGA Development Board dan menjalankan program untuk menampilkan penggunaan \textit{resource} dari proses yang sedang berjalan.

\begin{afigure}
    \lstinputlisting[frame=single, style=python]{images/programs/pid.py}
    \caption{Menampilkan PID sebuah proses dengan bahasa pemrograman Python.}
    \label{code:pid}
\end{afigure}

Pertama peneliti menampilkan ID proses dari Jupyter Notebook yang digunakan untuk mencatat penggunaan memory dan CPU dari penerapan filter menggunakan prosesor ARM dan FPGA. Cara menampilkan ID proses (PID) menggunakan bahasa pemrograman Python dapat dilihat pada gambar \ref{code:pid}. ID proses tersebut kemudian digunakan pada program \textbf{top} untuk menampilkan \textit{resource} yang digunakan oleh preses tersebut. 

\begin{afigure}
    \begin{lstlisting}[frame=single, style=shell] 
    $ top -d 0.1 -p 4382 -b >> arm-laplacian1.txt    
    \end{lstlisting}
    \caption{Menjalankan program \textbf{top} kemudian menyimpan hasilnya pada file arm-laplacian1.txt.}
    \label{code:top}
\end{afigure}

Pada gambar \ref{code:top} ditunjukan cara menggunakan program \textbf{top} untuk mencatat penggunaan \textit{resource} dari proses dengan ID 4382 kemudian \textit{output}nya disimpan pada file \textit{arm-laplacian1.txt}. Program \textbf{top} tersebut dijalankan sesaat sebelum proses penerapan filter dijalankan pada Jupyter Notebook. Output dari program top setiap 0.1 detik disimpan pada file text yang telah ditentukan. Setelah seluruh frame selesai difilter maka program \textbf{top} juga dihentikan.

\begin{afigure}
    \lstinputlisting[frame=single, style=plain]{images/programs/arm-laplacian1.txt}
    \caption{Potongan isi file arm-laplacian1.txt.}
    \label{code:arm-laplacian1.txt}
\end{afigure}

Isi file \textit{arm-laplacian1.txt} setelah program selesai dijalankan dapat dilihat pada gambar \ref{code:arm-laplacian1.txt}. Isi dari file hasil ini masih banyak mengandung informasi yang tidak dibutuhkan pada penelitian ini, sehingga perlu dilakukan proses ekstraksi informasi yang dibutuhkan saja. Kemudian file yang telah diekstraksi tersebut dibuat menjadi file CSV agar lebih mudah dalam proses pengolahan selanjutnya, seperti pada gambar \ref{code:arm-laplacian1.csv}. Proses tesebut dilakukan sebanyak 5 kali percobaan pada masing-masing kernel menggunakan prosesor ARM dan FPGA.

\begin{afigure}
    \lstinputlisting[frame=single, style=plain]{images/programs/arm-laplacian1.csv}
    \caption{Isi file arm-laplacian1.csv.}
    \label{code:arm-laplacian1.csv}
\end{afigure}


% Jelaskan Hasil
\section{Analisis Kinerja}
\subsection{Waktu Komuptasi}
% pengantar
% Dilakukan 5 kali percobaan dengan 50 frame dan 5 kali percobaan dengan 200 frame pada masing-masing kernel menggunakan ARM prosesor dan FPGA. 

% 50 FRAMe
Data waktu komputasi dengan menggunakan 50 frame pada maasing-masing kernel dapat dilihat pada tabel \ref{table:hasil-time50} dan grafik pada gambar \ref{fig:chart-time50}. Secara umum waktu komputasi dengan menggunakan prosesor ARM lebih lambat daripada waktu komputasi dengan menggunakan FPGA. Rata-rata waktu komputasi dengan prosesor ARM menggunakan 50 frame adalah 7,26 detik, sedangkan rata-rata waktu komputasi dengan FPGA menggunakan 50 frame hanya 0,82 detik.
\begin{atable}
    \caption{Tabel perbandingan waktu komputasi dengan menggunakan 50 frame.}
    \label{table:hasil-time50}
    \csvreader[
        head to column names,
        tabular=lcc,
        separator=semicolon,
        before table=\rowcolors{2}{gray!15}{gray!30},
        table head= \rowcolor{gray!50!black} 
            \color{white} Filter & 
            \color{white} Prosesor ARM (s) & 
            \color{white} FPGA (s)\\]
        {tables/hasil-time50.csv}
        {
            filter=\filter, 
            arm=\arm, 
            fpga=\fpga}
        {
            \filter & 
            \arm & 
            \fpga }
\end{atable}
\begin{figure}[H]
    \centering
    \includegraphics[width=0.81\linewidth, center]{images/chart/chart-time50.png}
    \caption{50 frame.}
    \label{fig:chart-time50}
    \caption{Grafik perbandingan waktu komputasi dengan 50 frame dan 200 frame}
\end{figure}

% 200 frame
Data waktu komputasi menggunakan 200 frame pada maasing-masing kernel dapat dilihat pada tabel \ref{table:hasil-time200} dan grafik pada gambar \ref{fig:chart-time200}. Rata-rata waktu komputasi dengan prosesor ARM menggunakan 200 frame adalah 29,06 detik, sedangkan rata-rata waktu komputasi dengan FPGA menggunakan 200 frame hanya 3,32 detik.
\begin{atable}
    \caption{Tabel perbandingan waktu komputasi dengan menggunakan 200 frame.}
    \label{table:hasil-time200}
    \csvreader[
        head to column names,
        tabular=lcc,
        separator=semicolon,
        before table=\rowcolors{2}{gray!15}{gray!30},
        table head= \rowcolor{gray!50!black} 
            \color{white} Filter & 
            \color{white} Prosesor ARM (s) & 
            \color{white} FPGA (s)\\]
        {tables/hasil-time200.csv}
        {
            filter=\filter, 
            arm=\arm, 
            fpga=\fpga}
        {
            \filter & 
            \arm & 
            \fpga }
\end{atable}

\begin{figure}[H]
    \centering
    \includegraphics[width=0.81\linewidth, center]{images/chart/chart-time200.png}
    \caption{200 frame.}
    \label{fig:chart-time200}
    \caption{Grafik perbandingan waktu komputasi dengan 50 frame dan 200 frame}
\end{figure}

Waktu komputasi tercepat dengan menggunakan prosesor ARM terdapat pada filter Laplacian yaitu 6,21 detik dengan 50 frame dan 24.85 detik dengan 200 frame. Sedangkan waktu komputasi paling lambat ketika menggunakan prosesor ARM terdapat pada filter Sharpening yaitu 8,04 detik dengan 50 frame dan 32,24 detik dengan 200 frame. 


\subsection{Frame Rate (FPS)}
Dengan mengetahui waktu komputasi dan jumlah frame maka frame rate atau FPS dapat dihitung menggunakan persamaan \ref{eq:fps}. Data FPS dari masing-masing kernel dengan prosesor ARM dan FPGA dapat dilihat pada tabel \ref{table:hasil-fps} dan grafik pada gambar \ref{fig:chart-fps}.
\begin{atable}
    \caption{Tabel perbandingan FPS dengan menggunakan prosesor ARM dan FPGA.}
    \label{table:hasil-fps}
    \csvreader[
        head to column names,
        tabular=lcc,
        separator=semicolon,
        before table=\rowcolors{2}{gray!15}{gray!30},
        table head= \rowcolor{gray!50!black} 
            \color{white} Filter & 
            \color{white} Prosesor ARM & 
            \color{white} FPGA\\]
        {tables/hasil-fps.csv}
        {
            filter=\filter, 
            arm=\arm, 
            fpga=\fpga}
        {
            \filter & 
            \arm & 
            \fpga }
\end{atable}
\begin{figure}[H]
    \includegraphics[width=0.81\linewidth, center]{images/chart/chart-fps.png}
    \caption{Grafik perbandingan FPS dengan menggunakan prosesor ARM dan FPGA.}
    \label{fig:chart-fps}
\end{figure}

Pada tabel \ref{table:hasil-fps} terlihat dengan menggunakan prosesor ARM diperoleh rata-rata 6.95 frame per detik (FPS), sedangkan ketika menggunakan FPGA diperoleh rata-rata 60.37 frame per detik. Terlihat pada grafik \ref{fig:chart-fps} nilai FPS dengan FPGA jauh lebih tinggi daripada dengan prosesor ARM yang ada pada FPGA Development Board.


\subsection{Penggunaan CPU}
Data perbandingan penggunaan CPU pada masing-masing kernel dengan prosesor ARM dan FPGA dapat dilihat pada tabel \ref{table:hasil-cpu} dan grafik pada gambar \ref{fig:chart-cpu}. Rata-rata penggunaan CPU dengan prosesor ARM adalah 99.58\% sedangkan dengan FPGA diperoleh 84.75\%. Data ini menunjukkan bahwa penggunaan CPU dengan prosesor ARM sedikit lebih besar daripada dengan FPGA.

\begin{atable}
    \caption{Tabel perbandingan penggunaan CPU dengan menggunakan prosesor ARM dan FPGA.}
    \label{table:hasil-cpu}
    \csvreader[
        head to column names,
        tabular=lcc,
        separator=semicolon,
        before table=\rowcolors{2}{gray!15}{gray!30},
        table head= \rowcolor{gray!50!black} 
            \color{white} Filter & 
            \color{white} Prosesor ARM (\%) & 
            \color{white} FPGA (\%)\\]
        {tables/hasil-cpu.csv}
        {
            filter=\filter, 
            arm=\arm, 
            fpga=\fpga}
        {
            \filter & 
            \arm & 
            \fpga }
\end{atable}
\begin{figure}[H]
    \includegraphics[width=0.81\linewidth, center]{images/chart/chart-cpu.png}
    \caption{Grafik perbandingan penggunaan CPU dengan menggunakan prosesor ARM dan FPGA.}
    \label{fig:chart-cpu}
\end{figure}

Penggunaan CPU terbesar dengan prosesor ARM yaitu pada kernel average blur (99,83\%) dan dengan FPGA pada kernel sobel horizontal (86,22\%). Penggunaan CPU terkecil dengan prosesor ARM yaitu pada kernel sharpening dan sobel horizontal (99,47\%) dan dengan FPGA pada kernel gaussian blur (83,70\%).


\subsection{Penggunaan Memory}
Data penggunaan memory dengan prosesor ARM dan FPGA dapat dilihat pada tabel \ref{table:hasil-mem} dan grafik pada gambar \ref{fig:chart-mem}. Data ini menunjukkan persentase memory yang digunakan pada masing-masing kernel. Rata-rata penggunaan memory dengan prosesor ARM adalah 25,37\% dan 24,86\% dengan FPGA. 
\begin{atable}
    \caption{Tabel perbandingan penggunaan memory dengan menggunakan prosesor ARM dan FPGA.}
    \label{table:hasil-mem}
    \csvreader[
        head to column names,
        tabular=lcc,
        separator=semicolon,
        before table=\rowcolors{2}{gray!15}{gray!30},
        table head= \rowcolor{gray!50!black} 
            \color{white} Filter & 
            \color{white} Prosesor ARM (\%) & 
            \color{white} FPGA (\%)\\]
        {tables/hasil-mem.csv}
        {
            filter=\filter, 
            arm=\arm, 
            fpga=\fpga}
        {
            \filter & 
            \arm & 
            \fpga }
\end{atable}
\begin{figure}[H]
    \includegraphics[width=0.81\linewidth, center]{images/chart/chart-mem.png}
    \caption{Grafik perbandingan penggunaan memory dengan menggunakan prosesor ARM dan FPGA.}
    \label{fig:chart-mem}
\end{figure}
Walaupun FPGA lebih baik daripada prosesor ARM pada segi waktu komputasi dan FPS namun penggunaan memory pada penerapan filter ini terlihat tidak jauh berbeda. Penggunaan memory FPGA ini hanya 0,51\% lebih rendah dari penggunaan memory dengan prosesor ARM.


\subsection{Resident Memory (RES)}
Data penggunaan resident memory atau RES dengan prosesor ARM dan FPGA dapat dilihat pada tabel \ref{table:hasil-res} dan grafik pada gambar \ref{fig:chart-res}. Data ini menunjukkan banyaknya RES (dalam satuan kilobyte) yang digunakan pada saat penerapan filter spasial pada video stream dengan masing-masing kernel.

\begin{atable}
    \caption{Tabel perbandingan penggunaan resident memory dengan menggunakan prosesor ARM dan FPGA.}
    \label{table:hasil-res}
    \csvreader[
        head to column names,
        tabular=lcc,
        separator=semicolon,
        before table=\rowcolors{2}{gray!15}{gray!30},
        table head= \rowcolor{gray!50!black} 
            \color{white} Filter & 
            \color{white} Prosesor ARM (KiB) & 
            \color{white} FPGA (KiB)\\]
        {tables/hasil-res.csv}
        {
            filter=\filter, 
            arm=\arm, 
            fpga=\fpga}
        {
            \filter & 
            \arm & 
            \fpga }
\end{atable}
\begin{figure}[H]
    \includegraphics[width=0.81\linewidth, center]{images/chart/chart-res.png}
    \caption{Grafik perbandingan penggunaan resident memory dengan menggunakan prosesor ARM dan FPGA.}
    \label{fig:chart-res}
\end{figure}
Rata-rata RES yang digunakan pada prosesor ARM adalah 129108,60 KiB dan 126437,20 KiB pada FPGA. Terlihat bahwa penggunaan RES pada prosesor ARM dan FPGA juga tidak jauh berbeda. Penggunaan RES terbesar dengan prosesor ARM yaitu pada kernel gaussian blur 131804,40 KiB, sedangkan dengan FPGA yaitu pada kernel sharpening 127931,20 KiB.


\subsection{Shared Memory (SHR)}
Data penggunaan shared memory dengan prosesor ARM dan FPGA dapat dilihat pada tabel \ref{table:hasil-shr} dan grafik pada gambar \ref{fig:chart-shr}. Data ini menunjukkan banyaknya shared memory (dalam satuan kilobyte) yang digunakan pada saat penerapan filter spasial pada video stream dengan masing-masing kernel.
\begin{atable}
    \caption{Tabel perbandingan penggunaan shared memory dengan menggunakan prosesor ARM dan FPGA.}
    \label{table:hasil-shr}
    \csvreader[
        head to column names,
        tabular=lcc,
        separator=semicolon,
        before table=\rowcolors{2}{gray!15}{gray!30},
        table head= \rowcolor{gray!50!black} 
            \color{white} Filter & 
            \color{white} Prosesor ARM (KiB) & 
            \color{white} FPGA (KiB)\\]
        {tables/hasil-shr.csv}
        {
            filter=\filter, 
            arm=\arm, 
            fpga=\fpga}
        {
            \filter & 
            \arm & 
            \fpga }
\end{atable}
\begin{figure}[H]
    \includegraphics[width=0.81\linewidth, center]{images/chart/chart-shr.png}
    \caption{Grafik perbandingan penggunaan shared memory dengan menggunakan prosesor ARM dan FPGA.}
    \label{fig:chart-shr}
\end{figure}
Rata-rata penggunaan shared memory pada prosesor ARM adalah 56325,57 KiB dan 54030,80 KiB pada FPGA. Terlihat bahwa penggunaan shared memory pada prosesor ARM sedikit lebih besar daripada FPGA. Penggunaan shared memory terbesar pada prosesor ARM yaitu pada kernel gaussian blur 56503,60 KiB dan kernel laplacian 55528,80 KiB pada FPGA. Penggunaan shared memory terkecil pada prosesor ARM yaitu pada kernel average blur 56157,40 KiB dan kernel laplacian 42568 KiB pada FPGA.

\subsection{Virtual Memory (VIRT)}
Data penggunaan virtual memory atau VIRT dapat dilihat pada tabel \ref{table:hasil-virt} dan grafik pada gambar \ref{fig:chart-virt}. Data ini menunjukkan banyaknya virtual memory (dalam satuan kilobyte) yang digunakan pada saat penerapan filter spasial pada video stream dengan masing-masing kernel.
\begin{atable}
    \caption{Tabel perbandingan penggunaan virtual memory dengan menggunakan prosesor ARM dan FPGA.}
    \label{table:hasil-virt}
    \csvreader[
        head to column names,
        tabular=lcc,
        separator=semicolon,
        before table=\rowcolors{2}{gray!15}{gray!30},
        table head= \rowcolor{gray!50!black} 
            \color{white} Filter & 
            \color{white} Prosesor ARM (KiB) & 
            \color{white} FPGA (KiB)\\]
        {tables/hasil-virt.csv}
        {
            filter=\filter, 
            arm=\arm, 
            fpga=\fpga}
        {
            \filter & 
            \arm & 
            \fpga }
\end{atable}
\begin{figure}[H]
    \includegraphics[width=0.81\linewidth, center]{images/chart/chart-virt.png}
    \caption{Grafik perbandingan penggunaan virtual memory dengan menggunakan prosesor ARM dan FPGA.}
    \label{fig:chart-virt}
\end{figure}
Rata-rata penggunaan VIRT pada prosesor ARM adalah 395407,64 KiB dan 395501,33 KiB pada FPGA. Rata-rata penggunaan VIRT pada FPGA sedikit lebih tinggi dari pada prosesor ARM. Penggunaan VIRT terbesar pada prosesor ARM yaitu pada kernel gaussian blur 398862,80 KiB dan kernel laplacian 395638,40 KiB. Penggunaan VIRT terkecil pada prosesor ARM yaitu pada kernel 393544,80 KiB dan kernel sobel vertical 395385,60 KiB pada FPGA.
