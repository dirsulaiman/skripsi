
\chapter{HASIL DAN PEMBAHASAN}


%Jelaskan Tentang Proses Filter spasial 
\section{Filter Spasial Linear}
\blindtext


% Jelaskan Proses Implementasi 
\section{Implementasi pada FPGA Development Board}
\blindtext
\subsection{Representasi Video Stream sebagai Citra Digital}
\subsection{Konversi Frame menjadi Grayscale}
\subsection{Penerapan Filter Spasial}
\subsection{Analisis Kinerja}


% Jelaskan Hasil
\section{Analisis Kinerja}
\subsection{Frame Rate (FPS)}

\begin{atable}
    \caption{Perbandingan waktu komputasi}
    \label{table:hasil-fps}
    \csvreader[
        head to column names,
        tabular=lcc,
        before table=\rowcolors{2}{gray!15}{gray!30},
        table head= \rowcolor{gray!50!black} 
            \color{white} Filter & 
            \color{white} Tanpa FPGA & 
            \color{white} Dengan FPGA \\]
        {tables/hasil-fps.csv}
        {filter=\filter, tanpafpga=\tanpafpga, denganfpga=\denganfpga}
        {\filter & \tanpafpga & \denganfpga }
\end{atable}

\subsubsection{Prosesor ARM}
\subsubsection{FPGA}

\subsection{Penggunaan CPU}
\blindtext
\subsubsection{Prosesor ARM}
\subsubsection{FPGA}

\subsection{Penggunaan Memory}
\blindtext
\subsubsection{Prosesor ARM}
\subsubsection{FPGA}

\subsection{Resident Memory (RES)}
\blindtext
\subsubsection{Prosesor ARM}
\subsubsection{FPGA}

\subsection{Shared Memory (SHR)}
\blindtext
\subsubsection{Prosesor ARM}
\subsubsection{FPGA}

\subsection{Virtual Memory (VIRT)}
\blindtext
\subsubsection{Prosesor ARM}
\subsubsection{FPGA}