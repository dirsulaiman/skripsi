
\chapter{HASIL DAN PEMBAHASAN}


\section{Landasan Teori}

\subsection{Citra Digital}
Citra digital dapat didefinisikan sebagai fungsi \textit{f(x,y)} berukuran M baris dan N kolom, dengan \textit{x} dan \textit{y} adalah kordinat spasial, dan amplitudo \textit{f} di titik kordinat (x,y) dinamakan intensitas atau tingkat keabuan dari citra pada citra tersebut \thecite{book:darma}. Pada umumnya warna dasar dalam citra RGB menggunakan penyimpanan 8 bit untuk menyimpan data warna, yang berarti setiap warna mempunyai gradasi sebanyak 255 warna . Dewasa ini, citra digital dapat menggunakan 16 bit untuk menyimpan data warna dasarnya, hal ini menyebabkan semakin banyak 