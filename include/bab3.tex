
\chapter{METODE PENENILITIAN}

\section{Waktu dan Lokasi Penelitian}
Penelitian ini dilaksanakan dari bulan Juni 2020 sampai dengan bulan Agustus 2020. Lokasi penelitian dilakukan di Laboratorium Rekayasa Perangkat Lunak Fakultas Matematika dan Ilmu Pengetahuan Alam, Universitas Hasanuddin Makassar.

\section{Tahapan Penelitian}

\section{Rancangan Sistem}

\section{Instrumen Penelitian}
Instrumen penelitian ini yaitu:
\begin{enumerate}[topsep=0pt,itemsep=0pt,partopsep=0pt, parsep=0pt]
    \item Kebutuhan perangkat lunak:
    \begin{enumerate}[topsep=0pt,itemsep=0pt,partopsep=0pt, parsep=0pt, label=\textbf{\alph*.}]
        \item Ubuntu 16 kayaknya 
        \item Python 3.6 kayaknya
        \item Jupyter Notebook
        \item Chrome Browser 
    \end{enumerate}
    \item Kebutuhan perangkat keras:
    \begin{enumerate}[topsep=0pt,itemsep=0pt,partopsep=0pt, parsep=0pt, label=\textbf{\alph*.}]
        \item FPGA Board Xilinx PYNQ-Z2
        \item Micro SD Card 8Gb (sebagai media penyimpanan OS pada FPGA Board)
        \item Monitor Eksternal (untuk menampilkan hasil filter dari FPGA)
        \item Laptop Lenovo Ideapad 320 (sebagai source video stream)
        \item 2 buah kabel HDMI (untuk HDMI input dan HDMI output pada FPGA) 
    \end{enumerate}
\end{enumerate}
