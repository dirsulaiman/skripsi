\chapter*{ABSTRAK}

Berbagai macam akselerator telah dikembangkan dalam meningkatkan kinerja dan efisiensi energi untuk menangani komputasi berat, salah satu diantaranya yaitu FPGA. FPGA mampu menangani beban komputasi yang begitu berat sehingga dapat digunakan untuk \textit{Digital Signal Processing}, \textit{Image Processing}, \textit{Neural Network}, dan sebagainya.
% metode
Pada penelitian ini penulis mencoba mengkaji kinerja yang dimiliki ARM prosesor dan FPGA pada FPGA \textit{Development Board} Xilinx PYNQ Z2 dalam penerapan filter spasial linear pada video \textit{stream}. Kernel filter yang digunakan pada penelitian ini yaitu \textit{average blur}, \textit{gaussian blur}, \textit{laplacian}, \textit{sharpen}, \textit{sobel horizontal} dan \textit{sobel vertical}. Parameter yang digunakan untuk mengukur kinerja ARM prosesor dan FPGA yaitu waktu komputasi, \textit{frame rate} (FPS), penggunaan CPU, penggunaan \textit{memory}, \textit{resident memory} (RES), \textit{shared memory} (SHR) dan \textit{virtual memory} (VIRT).
% hasil
Rata-rata waktu komputasi yang dibutuhkan untuk menerapkan filter spasial linear pada 200 frame dengan ARM prosesor adalah 29.06 detik sedangkan dengan FPGA rata-rata hanya dibutuhkan 3.32 detik. Waktu komputasi dengan FPGA 88.85\% lebih baik dibandingkan dengan ARM prosesor. Video hasil filter dengan ARM prosesor memperoleh rata-rata 6.95 fps sedangkan dengan FPGA rata-rata 60.37 fps. FPS dengan FPGA 88.49\% lebih baik lebih baik dibandingkan ARM prosesor. Penggunaan CPU pada FPGA 14.89\% lebih baik, penggunaan \textit{memory} pada FPGA 2.02\% lebih baik, penggunaan \textit{resident memory} 2.07\% lebih baik, dan penggunaan \textit{shared memory} 4.08\% lebih baik dibandingkan dengan ARM prosesor. Sedangkan penggunaan \textit{virtual memory} pada ARM prosesor 0.03\% lebih baik dibandingkan FPGA.


\begin{table}[h]
    \begin{tabular}{ p{0.17\textwidth} p{0.8\textwidth} }
        \\
        \textbf{Kata Kunci :} & filter spasial linear, FPGA, ARM prosesor, video \textit{stream}, \textit{video processing}
    \end{tabular}
\end{table}

 