\chapter{PENDAHULUAN}

\section{Latar Belakang}
% Tentang Citra digital
Citra digital merupakan citra yang dihasilkan dari pengolahan secara digital dengan merepresentasikan citra secara numerik dengan nilai-nilai diskrit. Suatu citra digital dapat direpresentasikan dalam bentuk matriks dengan fungsi \textit{f(x,y)} yang terdiri dari M kolom dan N baris. Perpotongan antara baris dan kolom disebut sebagai piksel (\textit{pixel}) \thecite{book:gonzalez}. Setiap piksel mewakili sebuah warna, pada citra biner sebuah piksel hanya berwarna hitam atau putih saja. Pada citra grayscale warna sebuah piksel mewakili tingkat keabuannya. Sedangkan pada citra warna (RGB) setiap piksel mewakili warna yang merupakan kombinasi dari tiga warna dasar ,yaitu merah, hijau dan biru. Pada umumnya warna dasar dalam citra RGB menggunakan penyimpanan 8 bit untuk menyimpan data warna atau tingkat keabuannya, yang berarti setiap warna mempunyai gradasi sebanyak 255 warna . Dewasa ini, citra digital dapat menggunakan 16 bit untuk menyimpan data warna atau tingkat keabuannya, hal ini menyebabkan semakin banyak gradasi warnanya sehingga citra yang dihasilkan memiliki tingkat warna yang jauh lebih banyak. Namun tentu saja hal ini mengakibatkan ukuran file citra digital yang dihasilkan juga menjadi semakin besar.

% Pengolahan Citra digital
Pengolahan citra digital merupakan proses mengolah piksel di dalam citra secara digital untuk tujuan tertentu. Berdasarkan tingkat pemrosesannya pengolahan citra digital dikelompokkan menjadi tiga kategori, yaitu: \textit{low-level}, \textit{mid-level} dan pemrosesan \textit{high-level}. Pemrosesan \textit{low-level} dilakukan dengan operasi primitif seperti \textit{image preprocessing} untuk mengurangi derau (\textit{noise}), memperbaiki kontras citra dan mempertajam citra (\textit{sharpening}). Pemrosesan \textit{mid-level} melibatkan tugas-tugas seperti segmentasi atau mempartisi gambar menjadi beberapa bagian atau objek, deskripsi objek untuk dilakukan pemrosesan lanjutan, dan klasifikasi objek yang terdapat dalam citra digital. Pemrosesan \textit{high-level} merupakan proses tingkat lanjut dari dua proses sebelumnya, dilakukan untuk mendapat informasi lebih yang terkandung dalam citra seperti \textit{pattern recognition}, \textit{template matching}, \textit{image analysis} dan sebagainya \thecite{book:gonzalez}.


% Filter Spasial
Konsep filter spasial pada pengolahan citra digital berasal dari penerapan transformasi Fourier untuk pemrosesan sinyal pada domain frekuensi. Istilah filter spasial ini digunakan untuk membedakan proses ini dengan filter pada domain frequensi. Proses filter dilakukan dengan cara menggeser filter kernel dari titik ke titik dalam citra digital. Istilah \textit{mask}, \textit{kernel}, \textit{template}, dan \textit{window} merupakan isitilah yang sama dan sering digunakan dalam pengolahan citra digital. Dalam penelitian ini penulis menggunakan istilah kernel untuk istilah tersebut. Konsep filter spasial linear mirip seperti konsep konvolusi pada domain frekuensi, dengan alasan tersebut filter spasial linear biasa disebut juga konvolusi sebuah kernel dengan citra digital \thecite{book:gonzalez}. Proses filter dalam pengolahan citra digital dilakukan dengan memanipulasi sebuah citra menggunakan kernel untuk menghasilkan citra yang baru, sehingga dengan kernel yang berbeda maka citra hasil yang didapat juga akan berbeda. 


% Stream Video adalah
Video stream dapat dipandang sebagai serangkaian citra digital berturut-turut \thecite{thesis:jin}. Berbeda dengan format video lainya, video stream ini tidak disimpan pada media penyimpanan sebagai file video melainkan setiap \textit{frame} langsung disalurkan dari sumber (\textit{source}) ke penerima, dalam hal ini FPGA. Dengan menganggap Video stream adalah kumpulan citra digital (\textit{frame}) maka dapat dilakukan metode pengolahan seperti pada citra digital, termasuk filter spasial. Setiap citra yang ditangkap dari source disebut sebagai \textit{frame}, setiap \textit{frame} ini dilakukan metode filter spasial kemudian hasilnya ditampilkan secara berkesinambungan sehingga tampak seperti video yang telah difilter.

Frame per second (\textit{fps}) atau \textit{frame rate} adalah banyaknya \textit{frame} yang ditampilkan per detik. Semakin tinggi \textit{fps} sebuah video maka semakin halus pula gerakan yang dapat ditampilkan karena dibentuk dari \textit{frame} yang lebih banyak, namun dengan jumlah \textit{frame} yang lebih besar tentu dibutuhkan juga \textit{resource} yang lebih besar dalam pengolahan video tersebut \thecite{pdf:marcin}. 
% Dalam penelitian ini video stream yang digunakan dibatasi 30 \textit{fps} saja dengan resolusi 720p.


% FPGA sebagai alat untuk implementasi
Field Programmable Gate Arrays atau FPGA adalah perangkat semikonduktor yang berbasis \textit{matriks configurable logic block} (CLBs) yang terhubung melalui interkoneksi yang dapat diprogram. FPGA dapat diprogram ulang dengan aplikasi atau fungsi yang diinginkan setelah \textit{manufacturing}. Fitur ini yang membedakan FPGA dengan \textit{Application Specific Integrated Circuits} (ASICs), yang dibuat khusus untuk tugas tertentu saja \thecite{XILINX}.

FPGA Xilinx PYNQ-Z2 adalah FPGA \textit{Board} yang digunakan pada penelitian ini secara \textit{official} dapat menerima input video stream dengan resolusi 720p. Setiap \textit{frame} yang diterima dari \textit{source} akan dilakukan proses filter spasial, kemudian hasilnya disalurkan melalui HDMI output untuk kemudian ditampilkan. Video hasil filter spasial yang ditampilakan akan mengalami penurunan \textit{fps}, hal ini disebabkan adanya penambahan jeda waktu komputasi untuk proses filter yang dilakukan pada setiap \textit{frame}. Pada kesempatan ini penulis ingin melakukan "Implementasi Filter Spasial Linear pada Video \textit{Stream} menggunakan FPGA \textit{Hardware Accelerator}".

\section{Rumusan Masalah}
Adapun rumusan masalah dalam penelitian ini yaitu:
\begin{enumerate}[topsep=0pt,itemsep=0pt,partopsep=0pt, parsep=0pt]
    \item Bagaimana cara implementasi fiter spasial linear pada video stream menggunakan FPGA? 
    \item Bagaimana kinerja FPGA dalam mengimplementasikan fiter spasial linear pada video stream? 
\end{enumerate}

\section{Batasan Masalah}
Berikut ini merupakan beberapa batasan dalam penelitian ini.
\begin{enumerate}[topsep=0pt,itemsep=0pt,partopsep=0pt, parsep=0pt]
    \item Filter kernel yang digunakan berukuran 3x3.
    \item Video stream yang digunakan dalam penelitian ini beresolusi 720p.
    \item Setiap frame dari video stream diubah menjadi citra grayscale sebelum dilakukan penerapan filter spasial.
    \item FPGA \textit{Board} yang digunakan adalah Xilinx PYNQ-Z2 dengan processor 650MHz dual-core ARM Cortex-A9.
\end{enumerate}

\pagebreak

\section{Tujuan Penelitian}
Adapun tujuan dari penelitian ini yaitu:
\begin{enumerate}[topsep=0pt,itemsep=0pt,partopsep=0pt, parsep=0pt]
    \item Mampu melakukan implementasi fiter spasial linear pada video stream menggunakan FPGA.
    \item Mengetahui kinerja FPGA dalam mengimplementasikan fiter spasial linear pada video stream.
\end{enumerate}

\section{Manfaat Penelitian}
Hasil dari penelitian ini diharapkan dapat memberikan pemahaman tentang penerapan filter spasial pada video stream. Selain itu, penelitian ini juga diharapkan dapat menjadi rujukan untuk melihat kinerja FPGA dalam mengimplementasikan filter spasial linear pada video stream.
