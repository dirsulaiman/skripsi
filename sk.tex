%Template pembuatan Skripsi Statistika dengan class unhasskripsi.

\documentclass[skripsi]{unhasskripsi}

\usepackage{amssymb}
\usepackage{amsfonts}
\usepackage{amscd}
\usepackage{amstext}
\usepackage{amsmath}
\usepackage{amsthm}
\usepackage{listings}
\usepackage{color}
\usepackage{float}
\usepackage{enumerate}
\usepackage{multirow}

\usepackage[pdftex,bookmarks=true]{hyperref}

%Untuk setting penomoran teorema, lemma, definisi, dll..
\newtheorem{teorema}{Teorema}[section]
\newtheorem{lemma}[teorema]{Lemma}
\newtheorem{sifat}[teorema]{Sifat}
\newtheorem{akibat}[teorema]{Akibat}
\newtheorem{proposisi}[teorema]{Proposisi}
\newtheorem{definisi}[teorema]{Definisi}

\theoremstyle{definition}
\newtheorem{contoh}[teorema]{Contoh}
\newtheorem{algoritma}{Algoritma}[chapter]
\renewcommand{\thealgoritma}{\arabic{chapter}.\arabic{algoritma}}

\renewenvironment{proof}{\vspace{1ex}\noindent{\bf Bukti.}}{\hfill$\blacksquare$\newline}

%-----------------------------------------------------------------
%Disini awal masukan untuk data skripsi
%-----------------------------------------------------------------
\titleind{PENAKSIRAN PARAMETER MODEL REGRESI GAMMA DENGAN \textit{LINEAR ALGEBRA SYNDROM} YANG DAPAT MENGATASI KEBIASAN}

%\titleeng{THE USE OF GENERAL LINEAR GROUP ON RSA CRYPTOSYSTEM}

\fullname{AQIFAH NOERFITRI}

\idnum{H121 08 259}

\examdate{\hspace{0.4cm} Mei 2019}

\degree{Sarjana Sains}

\yearsubmit{2019}

\program{Statistika}

\dept{Statistika}

\firstsupervisor{Dr. Loeky Haryanto, MS, MSc, MAT}

\secondsupervisor{Drs. Khaeruddin, MSc}

\firstexaminer{Drs. Khaeruddin, MSc}

\secondexaminer{Drs. Aluysius Sutjijana, M.Sc.}

\thirdexaminer{Drs. Retantyo Wardoyo, M.Sc., Ph.D.}

%-----------------------------------------------------------------
%Disini akhir masukan untuk data skripsi
%-----------------------------------------------------------------

\begin{document}

\cover

\titlepageind

\approvalpage

%-----------------------------------------------------------------
%Halaman Persembahan
%-----------------------------------------------------------------
\acknowledment
\begin{flushright}
\Large\emph\cal{Karya sederhana ini penulis persembahkan \\
untuk Bapak dan Almh. Ibuku tercinta}
\end{flushright}
%-----------------------------------------------------------------
%Disini akhir masukan untuk muka skripsi
%-----------------------------------------------------------------

%-----------------------------------------------------------------
%Halaman Motto
%-----------------------------------------------------------------
\motto
\emph{Pelajarilah Ilmu, karena mempelajarinya karena Allah adalah khasyah, menuntutnya adalah ibadah, mempelajarinya adalah Tasbih, mencarinya adalah Jihad, mengajarkannya kepada orang yang tidak mengetahui adalah Shadaqah, menye\-rahkan kepada ahlinya adalah Taqarrub. Ilmu adalah teman dekat dalam kesendirian dan sahabat dalam kesunyian.}
\begin{flushright}
(Muadz bin Jabal Radhiyyallahu'anhu)
\end{flushright}
%-----------------------------------------------------------------
%Disini akhir masukan untuk Motto
%-----------------------------------------------------------------

%-----------------------------------------------------------------
%Disini awal masukan untuk Prakata
%-----------------------------------------------------------------
\preface
Alhamdulillahirobil'alamin syukur kehadirat Allah SWT atas limpahan rahmat serta hidayah-Nya kepada penulis atas terselesaikannya skripsi ini. Sholawat dan salam semoga senantiasa tercurah kepada junjungan, suri teladan yang mulia, Nabi Muhammad SAW yang telah memberikan tuntunan yang sangat bijaksana pada kehidupan umat manusia umumnya dan pada penulis khususnya.

Suatu hal yang luar biasa pastinya dapat menyelesaikan tugas akhir ini, de\-ngan perjuangan yang tidak mudah, membutuhkan keteguhan hati, kesabaran, dan keikhlasan sehingga tertuntaskan sudah tugas akhir ini. Naik turunnya iman seorang hamba pun sempat menghinggapi diri penulis, sehingga tersendat-sendat dalam penyelesaiannya. Alhamdulillah atas karunia-Nya di hati selalu menggugah untuk maju.

Terlepas dari itu semua, tak bisa dielakkan bahwa penyusunan tugas akhir ini tak bisa lepas dari berbagai pihak atas semangatnya, kebersamaannya, serta ke\-sediaannya untuk berbagi dan melepas sejenak kejenuhan di hati.

Penulis haturkan terima kasih yang sebesar-besarnya kepada pihak-pihak yang telah mencurahkan segenap tenaga, pikiran, dan semangatnya kepada penulis.

\begin{enumerate}
\item Almahumah Ibuku tercinta, yang selalu menjaga, mengajari, dan menyayangiku semenjak aku masih kecil hingga akhir hayat Ibu.
\item Bapakku tercinta, serta Kakak dan Adikku tersayang yang selalu memberikan dorongan semangat, do'a, dan motivasi tiada henti.
\item Bapak Dr. Chairil Anwar selaku Dekan Fakultas MIPA Universitas Gadjah Mada.
\item Bapak Prof. Dr.rer.nat Widodo, M.S. selaku Ketua Jurusan Matematika dan Ibu Dr.rer.nat. Lina Aryati, M.S., selaku Ketua Program Studi Matematika Fakultas MIPA Universitas Gadjah Mada.
\item Bapak Imam Solekhudin, S.Si., M.Si. dan Bapak Drs. Yusuf M.A.,Math. selaku dosen wali akademik penulis. Terimakasih atas segala pengarahan dan semangat yang selalu Bapak berikan selama penulis belajar di Fakultas MIPA Universitas Gadjah Mada.
\item Ibu Dra. Diah Junia Eksi Palupi, M.S. selaku dosen pembimbing skripsi. Terima kasih atas bimbingan, kesabaran, dan pengertian yang telah diberikan kepada penulis dari awal penyusunan sampai akhir selesainya skripsi ini. Mohon maaf jika selama ini banyak bersikap yang kurang berkenan di hati Ibu.
\item Seluruh Dosen di FMIPA UGM yang telah memberikan ilmu-ilmunya kepada penulis.
\item ...
\item ...
\item ...
\item ...
\item ...
\end{enumerate}

Banyak kesalahan pastinya dalam penulisan tugas akhir ini. Masukan, saran, dan kritik demi kemajuan, dan kesempurnaan tulisan ini, demi kemaslahatan yang membangun, demi kehidupan yang lebih sempurna dimasa yang akan datang sangat diharapkan oleh penulis.

Terakhir, hanya milik Allah SWT segala kesempurnaan. Terimakasih dan mohon maaf atas segala kekurangannya. Semoga berguna.

\vspace{0.8cm}

\begin{tabular}{p{7.5cm}c}
&Yogyakarta, 23 Agustus 2010\\
&\\
&\\
&Penulis
\end{tabular}
%-----------------------------------------------------------------
%Disini akhir masukan Prakata
%-----------------------------------------------------------------

%-----------------------------------------------------------------
%Setting daftar isi
%-----------------------------------------------------------------
\makeatletter
\renewcommand\l@chapter[2]{%
  \ifnum \c@tocdepth >\z@
    \addpenalty\@secpenalty
    \addvspace{0.75em \@plus\p@}%
    \setlength\@tempdima{1.5em}%
    \begingroup
      \parindent \z@ \rightskip \@pnumwidth
      \parfillskip -\@pnumwidth
      \leavevmode \bfseries
      \advance\leftskip\@tempdima
      \hskip -\leftskip
      #1\nobreak\
      \leaders\hbox{$\m@th\mkern \@dotsep mu\hbox{.}\mkern \@dotsep mu$}
     \hfil \nobreak\hb@xt@\@pnumwidth{\hss #2}\par
    \endgroup
  \fi}
\makeatother

\tableofcontents
\addcontentsline{toc}{chapter}{DAFTAR ISI}
%-----------------------------------------------------------------
%-----------------------------------------------------------------


\listoftables
\addcontentsline{toc}{chapter}{DAFTAR TABEL}

\listoffigures
\addcontentsline{toc}{chapter}{DAFTAR GAMBAR}

%Halaman Lambang  dan Singkatan
\lambang
\begin{tabular}{cp{10cm}}
  $x\in A$ & : $x$ anggota A\\
  $A\subseteq X$ & : A himpunan bagian (\textit{subset}) atau sama dengan X\\
  $\mathbb{N}$ & : himpunan semua asli\\
  $\mathbb{Z}$ & : himpunan semua bilangan bulat\\
  $\mathbb{Z}^{+}$ & : himpunan semua bilangan bulat positif\\
  $\mathbb{R}$ & :himpunan semua bilangan real\\
  $C^{n}_{r}$ & : $r-$kombinasi dari $n$ unsur yang berbeda\\
  $\blacksquare$ & : akhir suatu bukti\\
  $\qed$\hfill & : akhir suatu contoh\\
  $\rightarrow$ & : menuju\\
  $\displaystyle\sum_{i=1}^{n}{a_{i}}$ & : penjumlahan $a_{1}+a_{2}+\cdots + a_{n}$\\
  $\displaystyle\prod_{i=1}^{n}{a_{i}}$ & : perkalian $a_{1}\cdot a_{2}\cdot\cdots \cdot a_{n}$\\
  $p \Rightarrow q$ & : jika $p$ maka $q$\\
  $\Leftrightarrow$ & : jika dan hanya jika\\
  $x\leftarrow a$ & : nilai $a$ dimasukkan ke $x$
\end{tabular}


%-----------------------------------------------------------------
%Disini awal masukan Intisari
%-----------------------------------------------------------------
\begin{abstractind}
Sistem kripto RSA merupakan suatu cipher blok. Himpunan plainteks dan cipherteks pada sistem kripto RSA adalah $\mathbb{Z}_{n}$, dengan $n$ adalah suatu hasil kali dua bilangan prima ganjil yang berbeda. Dalam skripsi ini, akan dibahas tentang pengembangan sistem kripto tersebut, yaitu dengan melibatkan suatu general linear grup. Hal ini dilakukan de\-ngan cara mengkonstruksikan blok-blok plainteks atau cipherteks menjadi matriks-matriks invertibel berukuran \mbox{$k \ \mathsf{x} \ k$} dengan entri-entrinya adalah elemen $\mathbb{Z}_{n}$. Sistem kripto pengembangan dari RSA ini disebut RSA Elatrash.
\end{abstractind}
%-----------------------------------------------------------------
%Disini akhir masukan Intisari
%-----------------------------------------------------------------

%-----------------------------------------------------------------
%Disini awal masukan untuk Abstract
%-----------------------------------------------------------------
\begin{abstracteng}
The RSA cryptosystem is a block cipher. The plaintext and ciphertext sets of RSA $\ $cryptosystem are $\ \mathbb{Z}_{n}$, $\ $where $\ n\ $ is a product $\ $of $\ $two $\ $distinct $\ $odd primes.$\ $ This Bachelor Thesis discusses about its development, which involving a general linear group on RSA cryptosystem. We do this by constructing plaintext or ciphertext blocks to invertible matrices \mbox{$k \ \mathsf{x} \ k$} with entries are elements of the $\mathbb{Z}_{n}$. This development of the RSA cryptosystem is called RSA Elatrash Cryptosystem.
\end{abstracteng}
%-----------------------------------------------------------------
%Disini akhir masukan Abstract
%-----------------------------------------------------------------

%-----------------------------------------------------------------
%Disini awal masukan untuk Bab
%-----------------------------------------------------------------
\input{Bab1.tex}
\input{Bab2.tex}
\input{Bab3.tex}
\input{Bab4.tex}
\input{Bab5.tex}
\input{Bab6.tex}
%-----------------------------------------------------------------
%Disini akhir masukan Bab
%-----------------------------------------------------------------

%-----------------------------------------------------------------
%Disini awal masukan untuk Daftar Pustaka
%-----------------------------------------------------------------
\begin{thebibliography}{99}
\bibitem[Adkins (1992)]{adk}
Adkins, W. A., 1992, \emph{Algebra "An Approach via Module Theory"},  Springer-Verlag New York, Inc., USA.

\bibitem[Anton (2000)]{ant}
Anton, H., 2000, \emph{Elementary Linear Algebra}, Eight Edition,  John Wiley and Sons, Inc., New York.

\bibitem[Brown (1993)]{brown}
Brown, W. C., 1993, \emph{Matrices over Commutative Rings}, Marcel Dekker, Inc., New York.

\bibitem[Buchmann (2000)]{buc}
Buchmann, J. A., 2000, \emph{Introduction to Cryptography}, Springer-Verlag New York, Inc., USA.

\bibitem[xxx (2xxx)]{xxx}
...
\end{thebibliography}
%-----------------------------------------------------------------
%Disini akhir masukan Daftar Pustaka
%-----------------------------------------------------------------

%-----------------------------------------------------------------
%Disini awal masukan untuk Lampiran
%-----------------------------------------------------------------
\appendix
\chapter{SKRIP PROGRAM JAVA}
\lstset{language=java, frame=single}
\lstset{numbers=left, numberstyle=\tiny, stepnumber=1, numbersep=5pt, basicstyle=\footnotesize}
\lstset{frame=shadowbox, rulesepcolor=\color{black}}
\lstset{keywordstyle=\color{blue}}
\lstset{caption=}
\lstinputlisting{Main.java}



\chapter{TABEL KODE ASCII}
\begin{figure}[H]
\begin{center}
\includegraphics[width=14cm]{ASCIIreguler}
\end{center}
\end{figure}
\newpage
\begin{figure}[H]
\begin{center}
\includegraphics[width=14.5cm]{ASCIIextended}
\end{center}
\end{figure}
%-----------------------------------------------------------------
%Disini akhir masukan Lampiran
%-----------------------------------------------------------------

\end{document} 